\PassOptionsToPackage{unicode}{hyperref}
\documentclass[aspectratio=1610, 9pt]{beamer}

% Load packages you need here
\usepackage{polyglossia}
\setmainlanguage{english}

\usepackage{csquotes}
    

\usepackage{amsmath}
\usepackage{amssymb}
\usepackage{mathtools}

\usepackage{hyperref}
\usepackage{bookmark}
\usepackage[
  locale=UK,
  separate-uncertainty=true,
  per-mode=symbol-or-fraction,
]{siunitx}
\usepackage[
  backend=biber,   % use modern biber backend
  autolang=hyphen, % load hyphenation rules for if language of bibentry is not
  % german, has to be loaded with \setotherlanguages
  % in the references.bib use langid={en} for english sources
  sorting=none,
  ]{biblatex}
  \addbibresource{references.bib}  % the bib file to use
  \DefineBibliographyStrings{english}{andothers = {{et\,al\adddot}}}  % replace u.a. with et al.
  
  
% load the theme after all packages
\usetheme[
  showtotalframes, % show total number of frames in the footline
]{tudo}

% Put settings here, like
\unimathsetup{
  math-style=ISO,
  bold-style=ISO,
  nabla=upright,
  partial=upright,
  mathrm=sym,
}
\setbeamertemplate{caption}{\raggedright\insertcaption\par}

\title{Spintronic emitters in the terahertz regime}
\subtitle{Applied optical spectroscopy}
\author[M.~Koch]{Max Koch}
\institute[]{TU Dortmund \\  Fakultät Physik}
%\titlegraphic{\includegraphics[width=0.3\textwidth, angle=90]{images/setup.jpeg}}


\begin{document}

\maketitle

\begin{frame}{Outline}
  \begin{columns}
    \begin{column}{.2\textwidth}
    \end{column}
    \begin{column}{.8\textwidth}
      \tableofcontents
    \end{column}
  \end{columns}
\end{frame}


\section{Recap}

\begin{frame}{The Terahertz Gap}
  \subsection{The spectrum}
  \begin{center}
    \begin{figure}
      \includegraphics[width=0.7\textwidth]{images/spectrum.png}
      \caption{\textcolor{tugreen}{The electromagntic spectrum} from G. P. Williams, Rep. Prog. Phys, \textbf{69} (2005) .}
      \nocite{spectrum_pic}
    \end{figure}
  \end{center}
\end{frame}


\begin{frame}{Terahertz}
  \subsection{Applications for THz}
  \begin{center}
    \begin{minipage}[c]{0.5\linewidth}
      So why do we need terahertz radiation?
      \vspace{0.2in}
      \begin{itemize}
        \item medicine \nocite{THzgap_applications}
        \vspace{0.1in}
        \item security \nocite{thz_explosive_detec}
        \vspace{0.1in}
        \item data transmission \& saving \nocite{communication,datasaving}
        \vspace{0.1in}
        \item physics \nocite{wiki_book}
      \end{itemize}
    \end{minipage}
\end{center}
\end{frame}

\section{Introduction}
\begin{frame}{Introduction}
\begin{columns}
  \begin{column}{.3\textwidth}
    \subsection{Common emitters}
    {\Large Common emitters:}
    \normalsize
    \vspace{0.3in}
    \begin{itemize}
      \item Photo conductive antenna 
      \vspace{0.3in}
      \item{Nonlinear crystals}
    \end{itemize}
  \end{column}
  \begin{column}{.7\textwidth}
    \includegraphics[width=0.7\textwidth]{images/Aufbau.pdf}
  \vspace{0.3in}
    \includegraphics[width=0.7\textwidth]{images/2_11_30_20normalX.pdf}
  \end{column}
\end{columns}
\end{frame}

\begin{frame}{What are Spintronic emitters?}
  \begin{columns}
    \begin{column}{0.3\textwidth}
      \begin{itemize}
        \item Ferromagnetic Material (FM)
        \vspace{0.3in}
        \item Non Magnetic (NM)
        \vspace{0.3in}
        \item Magnetic field
      \end{itemize}
    \end{column}
    \begin{column}{.6\textwidth}
      \begin{figure}
      \includegraphics[width=\textwidth]{pics/emitter.png}
        \caption{\textcolor{tugreen}{THz spintronic emitters} from E. Th. Papaioannou, Nanophotonics, (2005) .}
        \nocite{gruytersbook}
      \end{figure}
    \end{column}
  \end{columns}
\end{frame}

\begin{frame}{How does it work?}
  \begin{center}
    \begin{figure}
    \caption{\textcolor{tugreen}{How and why do Spintronic Terahertz Emitters work?} from T. S. Seifert, THz group Berlin}
    \vspace{0.3in}
    \includegraphics[width=0.6\textwidth]{pics/quenching.png}
    \end{figure}
  \end{center}
  \begin{equation*}
    E_\text{THz} \propto \dot{M}(t)
  \end{equation*}
  \begin{center}
    \Large Change in magnetization \rightarrow electric field
  \end{center}
\end{frame}

\begin{frame}{But we need more fieldstrength}
  \begin{columns}
    \begin{column}{0.6\textwidth}
      \begin{figure}
      \includegraphics[width=\textwidth]{pics/FMNM.png}
      \nocite{arxiv}
      \end{figure}
    \end{column}
    \begin{column}{0.4\textwidth}
      \includegraphics[width=\textwidth]{pics/FM-vsdouble.png}
    \end{column}
  \end{columns}
  \vspace{0.3in}
  \small\textcolor{tugreen}{Laser-induced terahertz spin transport in magnetic nanostructures arises from the same force as ultrafast demagnetization} from R. Rouzegar, L. Brandt et. al., arXiv.
  \vspace{0.3in}
  \begin{center}
    \Large Stronger if we attach \textbf{NM}-layer
  \end{center}
\end{frame}
%%%%%%%%%%%%%%%%%%%%%%%%%%%%%%%%%%%%%%%%%%%%%%%%%%%%%%%%%%%%%%%%%%%%%%%%%%%%%%%%%%%%%%%%%%%%%%%%%%%%%


% WOHER KOMMT DER WINKEL BEIM SPIN HALL EFFEKT?


%%%%%%%%%%%%%%%%%%%%%%%%%%%%%%%%%%%%%%%%%%%%%%%%%%%%%%%%%%%%%%%%%%%%%%%%%%%%%%%%%%%%%%%%%%%%%%%%%%%
\section{Inverse Spin Hall effect}
\begin{frame}{Where does the current come frome?}
\begin{center}
\normalsize
\end{center}
\begin{columns}
  \begin{column}{0.5\textwidth}
  Spin Hall effect
  \includegraphics[width=0.8\textwidth]{pics/Spin_Hall.png}
  \end{column}
  \begin{column}{0.5\textwidth}
    Inverse Spin Hall effect
    \includegraphics[width=.8\textwidth]{pics/inv_spin_hall.png}
  \end{column}
\end{columns}
\vspace{.2in}
\small{\textcolor{tugreen}{How and why do Spintronic Terahertz Emitters work?} from T. S. Seifert, THz group Berlin}
\vspace{0.2in}
\begin{center}
  \Large Inverse Spin Hall effect generates current
\end{center}
\end{frame}

\section{Summary}
\begin{frame}{Summary}
  \begin{columns}
    \begin{column}{.3\textwidth}
      \Large An overlook:
      \normalsize
      \vspace{0.3in}
      \begin{itemize}
        \item FM with magnetization
        \vspace{0.2in}
        \item spin current $j_s$ through $\si{\femto\second}$-laser pulse
        \vspace{0.2in}
        \item spin current to charge current
      \end{itemize}
      \vspace{0.2in}
      \begin{equation*}
        j_c = γj_s
      \end{equation*}
      \begin{itemize}
        \item charge current generates THz-Field
      \end{itemize}
    \end{column}
    \begin{column}{.7\textwidth}
      \begin{center}
        \includegraphics[width=0.7\textwidth]{pics/emitter.png}
      \end{center}
    \end{column}
  \end{columns}
\end{frame}

\begin{frame}{Setup}
  \begin{center}
    \begin{columns}
      \begin{column}{0.3\textwidth}
        \begin{itemize}
          \item Just change emitter
          \vspace{0.3in}
          \item Apply B-Field
          \vspace{0.3in}
          \item Put $Si$-lens behind crystal
        \end{itemize}
      \end{column}
      \begin{column}{.6\textwidth}
        \begin{center}
          \begin{figure}
          \includegraphics[width=\textwidth]{pics/setup.png}
          \caption{\textcolor{tugreen}{THz spintronic emitters} from E. Th. Papaioannou, Nanophotonics, (2005) .}
          \end{figure}
        \end{center}
      \end{column}
    \end{columns}
  \end{center}
\end{frame}

\section{Application}
\begin{frame}{Two Layers are not the end}
  \begin{center}
  \begin{columns}
    \begin{column}{0.5\textwidth}
      \begin{center}
        \hspace{0.1in}
        \includegraphics[width=.6\textwidth]{pics/trilayer.png}
        \includegraphics[width=0.7\textwidth]{pics/aplitude.png}
        \nocite{seifert2016efficient}
      \end{center}
    \end{column}
    \begin{column}{.5\textwidth}
      \begin{center}
        \small\textcolor{tugreen}{Efficient metallic spintronic emitters of ultrabroadband terahertz radiation} from T. Seifert, S. Jaiswal et. al., Nat. Photon, (2016).
        \vspace{0.2in}
        \includegraphics[width=.6\textwidth]{pics/vergleich.png}
      \end{center}
    \end{column}
  \end{columns}
  \vspace{0.3in}
\end{center}
\end{frame}

\section{Advantages}
\subsection{Polarization}
\begin{frame}{Polarization}
  \begin{columns}
    \begin{column}{0.3\textwidth}
    \begin{itemize}
      \item Change in B-Field changes THz-Field Polarization
      \vspace{0.3in}
      \item No filter needed
      \vspace{0.3in}
      \item Easy change of THz-Field Polarization
    \end{itemize}
    \end{column}
  \begin{column}{.7\textwidth}
  \begin{center}
    \includegraphics[width=0.7\textwidth]{pics/Bfeld.png}
  \end{center}
    \vspace{0.2in}
    \hspace{0.2in}
    \small{\textcolor{tugreen}{THz spintronic emitters} from E. Th. Papaioannou, Nanophotonics, (2005) .}
  \end{column}
  \end{columns}
\end{frame}

\subsection{Broadband}
\begin{frame}{Broadband}
  \begin{columns}
    \begin{column}{.3\textwidth}
      \begin{itemize}
        \item Super Broadband Signal
        \vspace{0.3in}
        \item Achieved with $W/Co40 Fe40 B 20 /Pt\;(\SI{5.8}{\nano\meter})$
      \end{itemize}
    \end{column}
    \begin{column}{.7\textwidth}
      \small\textcolor{tugreen}{Efficient metallic spintronic emitters of ultrabroadband terahertz radiation} from T. Seifert, S. Jaiswal et. al., Nat. Photon, (2016).
      \vspace{0.1in}
      \begin{center}
      \includegraphics[width=0.5\textwidth]{pics/broadband_comnpar.png} 
      \includegraphics[width=0.7\textwidth]{pics/broadband.png}
      \end{center}
    \end{column}
  \end{columns}
\end{frame}


\section*{Conclusion}
\begin{frame}{Conclusion}
\begin{columns}
\begin{column}{0.5\textwidth}
  \begin{itemize}
    \item Easy to setup
    \vspace{0.2in}
    \item Cheap to produce
    \vspace{0.2in}
    \item High damage threshold
    \vspace{0.2in}    
    \item Easy change in Polarization
    \vspace{0.2in}
    \item Very Broadband (no phonon modes)
    \vspace{0.2in}
    \item No problems with phasematching
  \end{itemize}
\end{column}
\begin{column}{.5\textwidth}
\includegraphics[width=.8\textwidth]{pics/advantage.png}
\small\textcolor{tugreen}{Spintronic terahertz emitters: Status and prospects from a materials perspective} from C. Bull, S. M. Hewett et. al., APL Materials, (2021).
\end{column}
\end{columns}
\end{frame}

\section*{Thank you}
\begin{frame}{}
  \begin{center}
  \textbf{\textcolor{tugreen}{Thank you all for your attention!}}
  \end{center}
\end{frame}

\subsection*{Not wavelength dependent}
\begin{frame}{Only power matters}
\includegraphics[width=\textwidth]{pics/wavelength.png}
\end{frame}

\section{References }
\printbibliography

\end{document}

